% Options for packages loaded elsewhere
\PassOptionsToPackage{unicode}{hyperref}
\PassOptionsToPackage{hyphens}{url}
%
\documentclass[
]{article}
\usepackage{amsmath,amssymb}
\usepackage{iftex}
\ifPDFTeX
  \usepackage[T1]{fontenc}
  \usepackage[utf8]{inputenc}
  \usepackage{textcomp} % provide euro and other symbols
\else % if luatex or xetex
  \usepackage{unicode-math} % this also loads fontspec
  \defaultfontfeatures{Scale=MatchLowercase}
  \defaultfontfeatures[\rmfamily]{Ligatures=TeX,Scale=1}
\fi
\usepackage{lmodern}
\ifPDFTeX\else
  % xetex/luatex font selection
\fi
% Use upquote if available, for straight quotes in verbatim environments
\IfFileExists{upquote.sty}{\usepackage{upquote}}{}
\IfFileExists{microtype.sty}{% use microtype if available
  \usepackage[]{microtype}
  \UseMicrotypeSet[protrusion]{basicmath} % disable protrusion for tt fonts
}{}
\makeatletter
\@ifundefined{KOMAClassName}{% if non-KOMA class
  \IfFileExists{parskip.sty}{%
    \usepackage{parskip}
  }{% else
    \setlength{\parindent}{0pt}
    \setlength{\parskip}{6pt plus 2pt minus 1pt}}
}{% if KOMA class
  \KOMAoptions{parskip=half}}
\makeatother
\usepackage{xcolor}
\setlength{\emergencystretch}{3em} % prevent overfull lines
\providecommand{\tightlist}{%
  \setlength{\itemsep}{0pt}\setlength{\parskip}{0pt}}
\setcounter{secnumdepth}{-\maxdimen} % remove section numbering
\ifLuaTeX
  \usepackage{selnolig}  % disable illegal ligatures
\fi
\IfFileExists{bookmark.sty}{\usepackage{bookmark}}{\usepackage{hyperref}}
\IfFileExists{xurl.sty}{\usepackage{xurl}}{} % add URL line breaks if available
\urlstyle{same}
\hypersetup{
  pdftitle={【论文笔记】EmoGen Emotional Image Content Generation with Text-to-Image Diffusion Models},
  hidelinks,
  pdfcreator={LaTeX via pandoc}}

\title{【论文笔记】EmoGen Emotional Image Content Generation with
Text-to-Image Diffusion Models}
\author{}
\date{2024-10-02T17:18:23+08:00}

\begin{document}
\maketitle
{
\setcounter{tocdepth}{3}
\tableofcontents
}
\hypertarget{abstract}{%
\subsection{Abstract}\label{abstract}}

\begin{itemize}
\tightlist
\item
  这是一篇关于图像生成的论文
\item
  提出了Emotional Image Content Generation
  (EICG)用于生成semantic-clear和emotion-faithful的图片

  \begin{itemize}
  \tightlist
  \item
    提出了情感空间emotion space
  \item
    使用了一个映射网络将emotion space和CLIP space对齐
  \item
    使用了Attribute loss和emotion confidence
  \end{itemize}
\item
  使用了三种指标:emotion accuracy, semantic clarity and semantic
  diversity
\end{itemize}

\hypertarget{motivation}{%
\subsection{Motivation}\label{motivation}}

\begin{itemize}
\tightlist
\item
  在CV中,情感很重要,有很多相关应用
\item
  扩散模型的发展提供了text-to-image的有力工具,但是在生成一些抽象的东西(比如情感)上有困难
\item
  以往的解决方案不能准确、显著地生成带情绪的图片
\item
  不能只从颜色和样式的角度生成情感
\item
  Emoset提供了与情感相关的数据集支撑
\item
  提出EICG
\end{itemize}

\hypertarget{method}{%
\subsection{Method}\label{method}}

\hypertarget{emotion-space}{%
\subsubsection{Emotion space}\label{emotion-space}}

\begin{itemize}
\tightlist
\item
  相似的emotion的点聚集,不相似的远离
\item
  使用resnet-50作为捕捉emotion表示的网络结构,使用emoset进行监督学习
\item
  使用交叉熵作为loss函数{}
\item
  在推理阶段,每个emotion
  cluster都由从对应的高斯分布中随机抽取,保证了有效性和多样性
\end{itemize}

\begin{quote}
emotion space的思想有点类似MLP中的embedding
vector,是否只要单独预训练好了,其他模型也可以即插即用?
\end{quote}

\begin{quote}
如果输入数据同时包含多种复杂情感应该怎么办呢?目前的这个使用resnet的解决方案看起来是一个分类的方法,面对复杂的情感,是否可以选取top
k的类别?
\end{quote}

\hypertarget{mapping-network}{%
\subsubsection{Mapping Network}\label{mapping-network}}

\begin{itemize}
\tightlist
\item
  使用一个映射网络将emotion space转换到CLIP space
\item
  因为在emotion space中的点在CLIP
  space中可能是分散的,因此我们不能用线性变换,因此使用了MLP进行非线性变换
\item
  然后再通过CLIP transformer
\item
  最后通过全连接层转换到CLIP space
\item
  为了更好地利用CLIP空间的知识,后两者是冻结参数的
\item
  上面三个转换步骤合在一起称为mapping network,其实只有第一层MLP需要训练
\item
  映射后的结果输入到扩散模型后的U-net中进行下游任务
\end{itemize}

\hypertarget{loss}{%
\subsubsection{loss}\label{loss}}

{}是噪声,{}是去噪网络,{}表示表示对时间{}的潜在噪声

但是,只用LDM
loss是不够的,因为同样的情感的语义可能是多样的,而只使用LDM会使一个抽象的情感坍缩为一个具体的事物,使其丧失多样性

因此,基于Emoset,提出了{}

其中{}是余弦相似性

\hypertarget{confidence}{%
\subsubsection{confidence}\label{confidence}}

因为不是所有图片都有情感,所以我们可以动态调整LDM loss和attr loss的占比

其中,{}是emotion confidence

其中{}是emotion space中的emotion
vector,{}是输入图片,{}是第i个emotion,{}是属于这个attribute的图片数量

\end{document}
